% Options for packages loaded elsewhere
\PassOptionsToPackage{unicode}{hyperref}
\PassOptionsToPackage{hyphens}{url}
%
\documentclass[
]{book}
\usepackage{amsmath,amssymb}
\usepackage{iftex}
\ifPDFTeX
  \usepackage[T1]{fontenc}
  \usepackage[utf8]{inputenc}
  \usepackage{textcomp} % provide euro and other symbols
\else % if luatex or xetex
  \usepackage{unicode-math} % this also loads fontspec
  \defaultfontfeatures{Scale=MatchLowercase}
  \defaultfontfeatures[\rmfamily]{Ligatures=TeX,Scale=1}
\fi
\usepackage{lmodern}
\ifPDFTeX\else
  % xetex/luatex font selection
\fi
% Use upquote if available, for straight quotes in verbatim environments
\IfFileExists{upquote.sty}{\usepackage{upquote}}{}
\IfFileExists{microtype.sty}{% use microtype if available
  \usepackage[]{microtype}
  \UseMicrotypeSet[protrusion]{basicmath} % disable protrusion for tt fonts
}{}
\makeatletter
\@ifundefined{KOMAClassName}{% if non-KOMA class
  \IfFileExists{parskip.sty}{%
    \usepackage{parskip}
  }{% else
    \setlength{\parindent}{0pt}
    \setlength{\parskip}{6pt plus 2pt minus 1pt}}
}{% if KOMA class
  \KOMAoptions{parskip=half}}
\makeatother
\usepackage{xcolor}
\usepackage{color}
\usepackage{fancyvrb}
\newcommand{\VerbBar}{|}
\newcommand{\VERB}{\Verb[commandchars=\\\{\}]}
\DefineVerbatimEnvironment{Highlighting}{Verbatim}{commandchars=\\\{\}}
% Add ',fontsize=\small' for more characters per line
\usepackage{framed}
\definecolor{shadecolor}{RGB}{248,248,248}
\newenvironment{Shaded}{\begin{snugshade}}{\end{snugshade}}
\newcommand{\AlertTok}[1]{\textcolor[rgb]{0.94,0.16,0.16}{#1}}
\newcommand{\AnnotationTok}[1]{\textcolor[rgb]{0.56,0.35,0.01}{\textbf{\textit{#1}}}}
\newcommand{\AttributeTok}[1]{\textcolor[rgb]{0.13,0.29,0.53}{#1}}
\newcommand{\BaseNTok}[1]{\textcolor[rgb]{0.00,0.00,0.81}{#1}}
\newcommand{\BuiltInTok}[1]{#1}
\newcommand{\CharTok}[1]{\textcolor[rgb]{0.31,0.60,0.02}{#1}}
\newcommand{\CommentTok}[1]{\textcolor[rgb]{0.56,0.35,0.01}{\textit{#1}}}
\newcommand{\CommentVarTok}[1]{\textcolor[rgb]{0.56,0.35,0.01}{\textbf{\textit{#1}}}}
\newcommand{\ConstantTok}[1]{\textcolor[rgb]{0.56,0.35,0.01}{#1}}
\newcommand{\ControlFlowTok}[1]{\textcolor[rgb]{0.13,0.29,0.53}{\textbf{#1}}}
\newcommand{\DataTypeTok}[1]{\textcolor[rgb]{0.13,0.29,0.53}{#1}}
\newcommand{\DecValTok}[1]{\textcolor[rgb]{0.00,0.00,0.81}{#1}}
\newcommand{\DocumentationTok}[1]{\textcolor[rgb]{0.56,0.35,0.01}{\textbf{\textit{#1}}}}
\newcommand{\ErrorTok}[1]{\textcolor[rgb]{0.64,0.00,0.00}{\textbf{#1}}}
\newcommand{\ExtensionTok}[1]{#1}
\newcommand{\FloatTok}[1]{\textcolor[rgb]{0.00,0.00,0.81}{#1}}
\newcommand{\FunctionTok}[1]{\textcolor[rgb]{0.13,0.29,0.53}{\textbf{#1}}}
\newcommand{\ImportTok}[1]{#1}
\newcommand{\InformationTok}[1]{\textcolor[rgb]{0.56,0.35,0.01}{\textbf{\textit{#1}}}}
\newcommand{\KeywordTok}[1]{\textcolor[rgb]{0.13,0.29,0.53}{\textbf{#1}}}
\newcommand{\NormalTok}[1]{#1}
\newcommand{\OperatorTok}[1]{\textcolor[rgb]{0.81,0.36,0.00}{\textbf{#1}}}
\newcommand{\OtherTok}[1]{\textcolor[rgb]{0.56,0.35,0.01}{#1}}
\newcommand{\PreprocessorTok}[1]{\textcolor[rgb]{0.56,0.35,0.01}{\textit{#1}}}
\newcommand{\RegionMarkerTok}[1]{#1}
\newcommand{\SpecialCharTok}[1]{\textcolor[rgb]{0.81,0.36,0.00}{\textbf{#1}}}
\newcommand{\SpecialStringTok}[1]{\textcolor[rgb]{0.31,0.60,0.02}{#1}}
\newcommand{\StringTok}[1]{\textcolor[rgb]{0.31,0.60,0.02}{#1}}
\newcommand{\VariableTok}[1]{\textcolor[rgb]{0.00,0.00,0.00}{#1}}
\newcommand{\VerbatimStringTok}[1]{\textcolor[rgb]{0.31,0.60,0.02}{#1}}
\newcommand{\WarningTok}[1]{\textcolor[rgb]{0.56,0.35,0.01}{\textbf{\textit{#1}}}}
\usepackage{longtable,booktabs,array}
\usepackage{calc} % for calculating minipage widths
% Correct order of tables after \paragraph or \subparagraph
\usepackage{etoolbox}
\makeatletter
\patchcmd\longtable{\par}{\if@noskipsec\mbox{}\fi\par}{}{}
\makeatother
% Allow footnotes in longtable head/foot
\IfFileExists{footnotehyper.sty}{\usepackage{footnotehyper}}{\usepackage{footnote}}
\makesavenoteenv{longtable}
\usepackage{graphicx}
\makeatletter
\def\maxwidth{\ifdim\Gin@nat@width>\linewidth\linewidth\else\Gin@nat@width\fi}
\def\maxheight{\ifdim\Gin@nat@height>\textheight\textheight\else\Gin@nat@height\fi}
\makeatother
% Scale images if necessary, so that they will not overflow the page
% margins by default, and it is still possible to overwrite the defaults
% using explicit options in \includegraphics[width, height, ...]{}
\setkeys{Gin}{width=\maxwidth,height=\maxheight,keepaspectratio}
% Set default figure placement to htbp
\makeatletter
\def\fps@figure{htbp}
\makeatother
\setlength{\emergencystretch}{3em} % prevent overfull lines
\providecommand{\tightlist}{%
  \setlength{\itemsep}{0pt}\setlength{\parskip}{0pt}}
\setcounter{secnumdepth}{5}
\usepackage{booktabs}
\ifLuaTeX
  \usepackage{selnolig}  % disable illegal ligatures
\fi
\usepackage[]{natbib}
\bibliographystyle{plainnat}
\IfFileExists{bookmark.sty}{\usepackage{bookmark}}{\usepackage{hyperref}}
\IfFileExists{xurl.sty}{\usepackage{xurl}}{} % add URL line breaks if available
\urlstyle{same}
\hypersetup{
  pdftitle={Untitled},
  pdfauthor={John Doe},
  hidelinks,
  pdfcreator={LaTeX via pandoc}}

\title{Untitled}
\author{John Doe}
\date{2024-11-08}

\begin{document}
\maketitle

{
\setcounter{tocdepth}{1}
\tableofcontents
}
\hypertarget{reloc-age-project}{%
\chapter{Reloc-Age Project}\label{reloc-age-project}}

Welcome to the Reloc-Age Project, a comprehensive analysis of housing choices and relocation patterns among individuals aged 55 and above.

\textbf{Key Statistics}
- {[}Statistic 1{]}
- {[}Statistic 2{]}

Use the navigation links to explore:
- \href{about.html}{About the Project}
- \href{methodology.html}{Methodology}
- \href{findings.html}{Findings}

\hypertarget{about-the-project}{%
\chapter{About the Project}\label{about-the-project}}

\hypertarget{background}{%
\section{Background}\label{background}}

The Reloc-Age project investigates how housing choices and relocations impact active and healthy aging among individuals aged 55 and above. By analyzing patterns in housing decisions and their effects on health outcomes, the project aims to generate significant insights into the aging process.

\hypertarget{objectives}{%
\section{Objectives}\label{objectives}}

The primary objectives of the Reloc-Age project are:

\begin{itemize}
\item
  \textbf{Study Housing Choices and Relocation Patterns}: Examine trends over time and by age in housing types and tenures among the Swedish population aged 55+.
\item
  \textbf{Assess Health Outcomes}: Analyze how housing aspects and relocations affect future health outcomes, considering factors such as age, sex, civil status, country of origin, adverse health events, loss of a partner, socio-economic status, and neighborhood characteristics.
\item
  \textbf{Develop Housing Counseling Interventions}: Create and pilot a novel housing counseling intervention tailored for older adults considering relocation.
\item
  \textbf{Contribute to Theory Development}: Advance theoretical frameworks related to housing choices, relocation, and active aging.
\end{itemize}

\hypertarget{research-questions}{%
\section{Research Questions}\label{research-questions}}

The project addresses several key research questions:

\begin{enumerate}
\def\labelenumi{\arabic{enumi}.}
\item
  What are the trends over time and by age regarding housing types and tenures?
\item
  How do housing aspects and relocations affect future health outcomes?

  \begin{itemize}
  \item
    How are these patterns influenced by factors such as age, sex, civil status, country of origin, adverse health events, loss of a partner, socio-economic status, and neighborhood characteristics?
  \item
    Given equal propensity for relocation based on baseline demographic, socio-economic, and health characteristics, how do specific housing decisions impact future health outcomes?
  \end{itemize}
\item
  What are the effects of adverse health events on housing choices and relocation patterns?

  \begin{itemize}
  \item
    What are the short- and long-term effects?
  \item
    How do these effects differ between men and women, across different disease and/or disability profiles, civil status, country of origin, and socio-economic status?
  \end{itemize}
\item
  What aspects of housing and health predict:

  \begin{itemize}
  \item
    Relocation to different housing options in the ordinary housing stock
  \item
    Relocation to residential care facilities
  \item
    Remaining in the present dwelling
  \end{itemize}
\item
  How is the complex interaction between objective and perceived aspects of housing and social factors associated with active and healthy aging, and what are the characteristics and trajectories of such dynamics?
\item
  What housing attributes do older adults considering relocation find important, and to what extent, when making their decisions on housing preferences?
\item
  How do older adults considering relocation reason regarding:

  \begin{itemize}
  \item
    Different housing options
  \item
    Motives for considering and effectuating relocation
  \item
    To what extent are their motives fulfilled?
  \end{itemize}
\item
  Is the newly developed housing counseling intervention usable, feasible, and acceptable for the Swedish municipality context, and what are the pros and cons of different delivery formats?
\item
  Which outcomes should be used to investigate the effectiveness of housing counseling, and what are:

  \begin{itemize}
  \item
    The responsiveness and
  \item
    The intervention effects on the selected primary and secondary outcomes, as indicated by the results of the pilot study?
  \end{itemize}
\item
  What are the main concepts and pathways of a theory on housing choices, relocation, and active aging?
\end{enumerate}

\hypertarget{team}{%
\section{Team}\label{team}}

The Reloc-Age project is conducted by a multidisciplinary team of researchers specializing in gerontology, public health, sociology, and housing studies. The team's diverse expertise ensures a comprehensive approach to understanding the complexities of housing and aging.

\hypertarget{funding-and-acknowledgments}{%
\section{Funding and Acknowledgments}\label{funding-and-acknowledgments}}

This project is funded by {[}Funding Organization{]}, supporting research initiatives aimed at improving the quality of life for older adults. We extend our gratitude to all participants and collaborators who contribute to the success of this study.

\hypertarget{data-sources-and-provenance}{%
\chapter{Data Sources and Provenance}\label{data-sources-and-provenance}}

\hypertarget{overview}{%
\section{Overview}\label{overview}}

This project utilizes comprehensive register data from Statistics Sweden (SCB) to analyze housing choices and health outcomes among individuals aged 55 and above. The datasets encompass demographic information, health records, socioeconomic status, and housing details, providing a robust foundation for our research.

\hypertarget{data-source-summary}{%
\section{Data Source Summary}\label{data-source-summary}}

\begin{longtable}[]{@{}
  >{\raggedright\arraybackslash}p{(\columnwidth - 8\tabcolsep) * \real{0.1623}}
  >{\raggedright\arraybackslash}p{(\columnwidth - 8\tabcolsep) * \real{0.3701}}
  >{\raggedright\arraybackslash}p{(\columnwidth - 8\tabcolsep) * \real{0.0844}}
  >{\raggedright\arraybackslash}p{(\columnwidth - 8\tabcolsep) * \real{0.2597}}
  >{\raggedright\arraybackslash}p{(\columnwidth - 8\tabcolsep) * \real{0.1234}}@{}}
\toprule\noalign{}
\begin{minipage}[b]{\linewidth}\raggedright
Data Source
\end{minipage} & \begin{minipage}[b]{\linewidth}\raggedright
Description
\end{minipage} & \begin{minipage}[b]{\linewidth}\raggedright
Time Period
\end{minipage} & \begin{minipage}[b]{\linewidth}\raggedright
Key Variables
\end{minipage} & \begin{minipage}[b]{\linewidth}\raggedright
Data Provider
\end{minipage} \\
\midrule\noalign{}
\endhead
\bottomrule\noalign{}
\endlastfoot
Total Population Register & Contains demographic information for all residents & 1968--2022 & Personal identity number, birth date, sex, marital status, country of birth, citizenship, migration details & Statistics Sweden \\
Longitudinal Integration Database for Health Insurance and Labour Market Studies (LISA) & Integrates data on employment, income, and social benefits & 1990--2022 & Employment status, income levels, educational attainment, social benefits received & Statistics Sweden \\
National Patient Register & Records inpatient and outpatient care details & 1987--2022 & Diagnoses, treatments, hospital admissions and discharges & National Board of Health and Welfare \\
Cause of Death Register & Provides information on mortality and causes of death & 1961--2022 & Date of death, underlying and contributing causes of death & National Board of Health and Welfare \\
Dwelling Register & Contains data on housing and living conditions & 2012--2022 & Type of dwelling, ownership status, household composition & Statistics Sweden \\
\end{longtable}

Each data source contributes unique insights relevant to our research objectives, as described below.

\hypertarget{total-population-register}{%
\section{Total Population Register}\label{total-population-register}}

\begin{itemize}
\tightlist
\item
  \textbf{Description}: Provides comprehensive demographic information for all individuals registered in Sweden, including personal identity numbers, birth dates, sex, marital status, country of birth, citizenship, and migration details.
\item
  \textbf{Time Period}: 1968--2022
\item
  \textbf{Key Variables}:

  \begin{itemize}
  \tightlist
  \item
    Personal identity number
  \item
    Birth date
  \item
    Sex
  \item
    Marital status
  \item
    Country of birth
  \item
    Citizenship
  \item
    Migration details (immigration and emigration dates)
  \end{itemize}
\item
  \textbf{Data Provider}: Statistics Sweden
\item
  \textbf{Considerations}: This register is crucial for linking individuals across different datasets using personal identity numbers.
\end{itemize}

\hypertarget{longitudinal-integration-database-for-health-insurance-and-labour-market-studies-lisa}{%
\section{Longitudinal Integration Database for Health Insurance and Labour Market Studies (LISA)}\label{longitudinal-integration-database-for-health-insurance-and-labour-market-studies-lisa}}

\begin{itemize}
\tightlist
\item
  \textbf{Description}: Integrates data from various sources to provide information on employment, income, education, and social benefits for individuals aged 16 and above.
\item
  \textbf{Time Period}: 1990--2022
\item
  \textbf{Key Variables}:

  \begin{itemize}
  \tightlist
  \item
    Employment status
  \item
    Income levels
  \item
    Educational attainment
  \item
    Social benefits received
  \end{itemize}
\item
  \textbf{Data Provider}: Statistics Sweden
\item
  \textbf{Considerations}: LISA is updated annually and is essential for analyzing socioeconomic factors in relation to health and housing outcomes.
\end{itemize}

\hypertarget{national-patient-register}{%
\section{National Patient Register}\label{national-patient-register}}

\begin{itemize}
\tightlist
\item
  \textbf{Description}: Contains information on inpatient and outpatient care, including diagnoses, treatments, and hospital admissions and discharges.
\item
  \textbf{Time Period}: 1987--2022
\item
  \textbf{Key Variables}:

  \begin{itemize}
  \tightlist
  \item
    Diagnoses (ICD codes)
  \item
    Treatments and procedures
  \item
    Hospital admissions and discharge dates
  \end{itemize}
\item
  \textbf{Data Provider}: National Board of Health and Welfare
\item
  \textbf{Considerations}: The register is vital for analyzing health outcomes and their association with housing choices.
\end{itemize}

\hypertarget{cause-of-death-register}{%
\section{Cause of Death Register}\label{cause-of-death-register}}

\begin{itemize}
\tightlist
\item
  \textbf{Description}: Provides data on mortality, including dates and causes of death, for all deceased individuals registered in Sweden.
\item
  \textbf{Time Period}: 1961--2022
\item
  \textbf{Key Variables}:

  \begin{itemize}
  \tightlist
  \item
    Date of death
  \item
    Underlying cause of death
  \item
    Contributing causes of death
  \end{itemize}
\item
  \textbf{Data Provider}: National Board of Health and Welfare
\item
  \textbf{Considerations}: This register allows for the examination of mortality rates and causes in relation to housing and relocation patterns.
\end{itemize}

\hypertarget{dwelling-register}{%
\section{Dwelling Register}\label{dwelling-register}}

\begin{itemize}
\tightlist
\item
  \textbf{Description}: Contains detailed information on housing and living conditions, including type of dwelling, ownership status, and household composition.
\item
  \textbf{Time Period}: 2012--2022
\item
  \textbf{Key Variables}:

  \begin{itemize}
  \tightlist
  \item
    Type of dwelling (e.g., apartment, single-family house)
  \item
    Ownership status (owned, rented)
  \item
    Household composition (number of residents, family structure)
  \end{itemize}
\item
  \textbf{Data Provider}: Statistics Sweden
\item
  \textbf{Considerations}: This register is essential for analyzing housing choices and their impact on health outcomes.
\end{itemize}

\hypertarget{data-integration-and-linkage}{%
\chapter{Data Integration and Linkage}\label{data-integration-and-linkage}}

\hypertarget{overview-1}{%
\section{Overview}\label{overview-1}}

This project combines data from multiple register sources provided by Statistics Sweden (SCB) and the National Board of Health and Welfare. Integrating these datasets enables comprehensive analyses of housing choices, health outcomes, socioeconomic status, and demographic trends. The integration process primarily involves linking individual records across datasets using unique identifiers and ensuring data consistency across time and sources.

\hypertarget{linkage-methodology}{%
\section{Linkage Methodology}\label{linkage-methodology}}

Each dataset includes unique personal identifiers that allow for accurate matching across registers. The primary identifier used for linkage is the \textbf{Personal Identity Number (PIN)}, assigned to each resident in Sweden. This identifier enables the seamless integration of information from different datasets, ensuring that all data points accurately correspond to the same individual.

\hypertarget{steps-in-the-linkage-process}{%
\subsection{Steps in the Linkage Process}\label{steps-in-the-linkage-process}}

\begin{enumerate}
\def\labelenumi{\arabic{enumi}.}
\tightlist
\item
  \textbf{Data Standardization}: Each dataset was reviewed to ensure consistent formatting of the PIN and other relevant variables used in the linkage process.
\item
  \textbf{Primary Linkage Using PIN}: The Personal Identity Number serves as the primary key for linking data across registers.
\item
  \textbf{Secondary Checks}: In cases where additional verification was required, variables like \textbf{birth date} and \textbf{sex} were used to confirm the linkage.
\end{enumerate}

\hypertarget{example-linkage}{%
\subsection{Example Linkage}\label{example-linkage}}

An example of the linkage process would involve connecting records from the \textbf{Total Population Register} and the \textbf{National Patient Register}. Using the PIN, we link demographic data (e.g., birth date, sex, migration details) with health data (e.g., diagnoses, treatments) for the same individual. This linkage enables us to analyze health outcomes in the context of demographic factors.

\hypertarget{challenges-in-data-integration}{%
\section{Challenges in Data Integration}\label{challenges-in-data-integration}}

Integrating multiple large datasets poses several challenges, including:

\begin{itemize}
\tightlist
\item
  \textbf{Inconsistent Data Formats}: Some datasets required reformatting to ensure compatibility across registers. For example, date formats and categorical variables like employment status were standardized.
\item
  \textbf{Missing or Incomplete Data}: Some records in individual registers had missing values for certain variables, particularly for older data entries. Missing values were managed carefully to avoid biases in analyses.
\item
  \textbf{Changes Over Time}: Registers have undergone updates, including changes in variable definitions and data collection methods. These changes were documented and accounted for during integration to maintain data consistency.
\end{itemize}

\hypertarget{quality-control-measures}{%
\section{Quality Control Measures}\label{quality-control-measures}}

To ensure the accuracy of the integrated dataset, the following quality control measures were implemented:

\begin{enumerate}
\def\labelenumi{\arabic{enumi}.}
\tightlist
\item
  \textbf{Cross-Verification}: After linkage, a subset of records was cross-verified across multiple registers. For example, age, sex, and other demographic details were compared across sources to confirm consistency.
\item
  \textbf{Missing Data Checks}: Missing values were flagged and examined to assess their impact on analyses. Imputation methods were applied only where it was appropriate, and records with critical missing information were handled based on predefined rules.
\item
  \textbf{Duplicate Record Removal}: Each dataset was checked for duplicate entries based on the PIN and relevant timestamp information, ensuring only unique records per individual and time point.
\end{enumerate}

\hypertarget{data-processing-pipeline}{%
\section{Data Processing Pipeline}\label{data-processing-pipeline}}

The integration process follows a structured data processing pipeline:

\begin{enumerate}
\def\labelenumi{\arabic{enumi}.}
\tightlist
\item
  \textbf{Data Extraction}: Data is extracted from each register using standardized methods, ensuring that only necessary variables and records are included.
\item
  \textbf{Cleaning and Transformation}: Each dataset undergoes cleaning and transformation, including standardization of formats, handling of missing values, and filtering for relevant time periods.
\item
  \textbf{Linkage and Verification}: Datasets are linked using the PIN, with additional verification steps applied to ensure accuracy.
\item
  \textbf{Final Quality Assurance}: The integrated dataset undergoes a final quality check, including consistency checks and validation against known statistics (e.g., population demographics).
\end{enumerate}

\hypertarget{ethical-considerations}{%
\section{Ethical Considerations}\label{ethical-considerations}}

Due to the sensitive nature of the data, strict ethical guidelines were followed throughout the integration process. All personal identifiers were handled in compliance with data privacy regulations, and access to identifiable information was restricted to authorized personnel only.

\hypertarget{footnotes-and-citations}{%
\chapter{Footnotes and citations}\label{footnotes-and-citations}}

\hypertarget{footnotes}{%
\section{Footnotes}\label{footnotes}}

Footnotes are put inside the square brackets after a caret \texttt{\^{}{[}{]}}. Like this one \footnote{This is a footnote.}.

\hypertarget{citations}{%
\section{Citations}\label{citations}}

Reference items in your bibliography file(s) using \texttt{@key}.

For example, we are using the \textbf{bookdown} package \citep{R-bookdown} (check out the last code chunk in index.Rmd to see how this citation key was added) in this sample book, which was built on top of R Markdown and \textbf{knitr} \citep{xie2015} (this citation was added manually in an external file book.bib).
Note that the \texttt{.bib} files need to be listed in the index.Rmd with the YAML \texttt{bibliography} key.

The RStudio Visual Markdown Editor can also make it easier to insert citations: \url{https://rstudio.github.io/visual-markdown-editing/\#/citations}

\hypertarget{data-processing-and-quality-control}{%
\chapter{Data Processing and Quality Control}\label{data-processing-and-quality-control}}

\hypertarget{overview-2}{%
\section{Overview}\label{overview-2}}

After data integration, each dataset underwent a comprehensive data processing workflow to ensure consistency, accuracy, and usability. The data processing steps included cleaning, standardizing variables, handling missing data, and performing quality assurance checks. These measures enhance the reliability of analyses conducted using the integrated dataset.

\hypertarget{data-cleaning}{%
\section{Data Cleaning}\label{data-cleaning}}

Data cleaning was a critical step in preparing the data for analysis. The following procedures were applied:

\begin{enumerate}
\def\labelenumi{\arabic{enumi}.}
\tightlist
\item
  \textbf{Duplicate Removal}: Duplicates were identified and removed based on unique identifiers (e.g., Personal Identity Number) and timestamps, ensuring that each record is unique to an individual and time point.
\item
  \textbf{Outlier Detection and Management}: Statistical methods were used to identify outliers for continuous variables (e.g., income, age). Outliers were flagged and either corrected or removed based on predefined thresholds relevant to each variable.
\item
  \textbf{String Standardization}: Categorical variables (e.g., marital status, employment status) were standardized for uniform spelling, capitalization, and value labels.
\end{enumerate}

\hypertarget{example-income-outliers}{%
\subsection{Example: Income Outliers}\label{example-income-outliers}}

For income data, values beyond the 99th percentile were reviewed to detect potential outliers. If values were deemed implausible (e.g., extremely high or negative income), they were flagged for further examination or removal.

\hypertarget{variable-standardization}{%
\section{Variable Standardization}\label{variable-standardization}}

To ensure compatibility across data sources, key variables were standardized to maintain consistency in format and structure.

\begin{itemize}
\tightlist
\item
  \textbf{Date Formatting}: Dates were converted to a standard \texttt{YYYY-MM-DD} format to enable consistent temporal analyses.
\item
  \textbf{Categorical Variables}: Categorical variables such as gender, marital status, and employment type were recoded with consistent labels across datasets. For instance, gender was standardized to ``Male'' and ``Female'' for uniformity.
\item
  \textbf{Units of Measure}: Variables with measurable units (e.g., income in SEK) were verified for consistency, ensuring that all values are recorded in the same units.
\end{itemize}

\hypertarget{example-employment-status}{%
\subsection{Example: Employment Status}\label{example-employment-status}}

Employment status from different sources was standardized as follows:
- ``Employed full-time'' and ``Employed part-time'' were recoded as ``Employed.''
- ``Unemployed'' was uniformly labeled as ``Unemployed'' across all sources.

\hypertarget{handling-missing-data}{%
\section{Handling Missing Data}\label{handling-missing-data}}

Missing data were handled based on the type of variable, data source, and relevance to the analysis. The following strategies were applied:

\begin{enumerate}
\def\labelenumi{\arabic{enumi}.}
\tightlist
\item
  \textbf{Listwise Deletion}: If a record contained missing values in critical variables (e.g., PIN, birth date), the entire record was removed to maintain data integrity.
\item
  \textbf{Imputation for Select Variables}: Imputation methods, such as median imputation or last observation carried forward (LOCF), were applied for select continuous variables where missingness was minimal and imputation would not introduce bias.
\item
  \textbf{Flagging Missing Data}: For key variables with high missingness, flags were created to identify records with missing values, allowing analysts to assess the impact of missing data on specific analyses.
\end{enumerate}

\hypertarget{example-age-imputation}{%
\subsection{Example: Age Imputation}\label{example-age-imputation}}

For records where age was missing but other demographic details were available, median imputation based on gender and region was used. For instance, if a participant's age was missing, the median age for their gender and region was substituted.

\hypertarget{quality-control-measures-1}{%
\section{Quality Control Measures}\label{quality-control-measures-1}}

To ensure the reliability and consistency of the processed data, the following quality control measures were implemented:

\begin{enumerate}
\def\labelenumi{\arabic{enumi}.}
\tightlist
\item
  \textbf{Cross-Verification with External Data}: The processed data were cross-verified with national statistics to validate accuracy. For example, demographic distributions (e.g., age and gender) were compared against official records.
\item
  \textbf{Internal Consistency Checks}: Consistency checks were conducted to identify and resolve discrepancies within the data. For instance, age and birth year were cross-checked for compatibility.
\item
  \textbf{Random Spot Checks}: A random sample of records was reviewed to ensure data accuracy and completeness across different variables. This included manually reviewing records for logical consistency and comparing raw data with processed output.
\item
  \textbf{Automated Quality Reports}: Automated reports were generated to summarize missing values, outliers, and any inconsistencies found during processing. These reports provided an overview of data quality and highlighted areas that required additional attention.
\end{enumerate}

\hypertarget{data-transformation-pipeline}{%
\section{Data Transformation Pipeline}\label{data-transformation-pipeline}}

The data processing and quality control were organized within a structured pipeline, including the following stages:

\begin{enumerate}
\def\labelenumi{\arabic{enumi}.}
\tightlist
\item
  \textbf{Data Import}: Raw data were imported from each register, ensuring that all variables were accurately captured and labeled.
\item
  \textbf{Data Cleaning and Standardization}: Each dataset was cleaned and standardized according to the procedures outlined above.
\item
  \textbf{Linkage and Verification}: After cleaning, datasets were linked by Personal Identity Number, with additional verification steps to confirm accuracy.
\item
  \textbf{Final Quality Control}: A final quality control stage included summary reports and consistency checks, ensuring that the processed data was ready for analysis.
\end{enumerate}

\hypertarget{ethical-considerations-1}{%
\section{Ethical Considerations}\label{ethical-considerations-1}}

Throughout the data processing steps, strict adherence to ethical guidelines was maintained. Personal identifiers were handled in a secure environment, and only authorized personnel had access to sensitive data. All processing steps were documented to ensure transparency and reproducibility.

\hypertarget{sharing-your-book}{%
\chapter{Sharing your book}\label{sharing-your-book}}

\hypertarget{publishing}{%
\section{Publishing}\label{publishing}}

HTML books can be published online, see: \url{https://bookdown.org/yihui/bookdown/publishing.html}

\hypertarget{pages}{%
\section{404 pages}\label{pages}}

By default, users will be directed to a 404 page if they try to access a webpage that cannot be found. If you'd like to customize your 404 page instead of using the default, you may add either a \texttt{\_404.Rmd} or \texttt{\_404.md} file to your project root and use code and/or Markdown syntax.

\hypertarget{metadata-for-sharing}{%
\section{Metadata for sharing}\label{metadata-for-sharing}}

Bookdown HTML books will provide HTML metadata for social sharing on platforms like Twitter, Facebook, and LinkedIn, using information you provide in the \texttt{index.Rmd} YAML. To setup, set the \texttt{url} for your book and the path to your \texttt{cover-image} file. Your book's \texttt{title} and \texttt{description} are also used.

This \texttt{gitbook} uses the same social sharing data across all chapters in your book- all links shared will look the same.

Specify your book's source repository on GitHub using the \texttt{edit} key under the configuration options in the \texttt{\_output.yml} file, which allows users to suggest an edit by linking to a chapter's source file.

Read more about the features of this output format here:

\url{https://pkgs.rstudio.com/bookdown/reference/gitbook.html}

Or use:

\begin{Shaded}
\begin{Highlighting}[]
\NormalTok{?bookdown}\SpecialCharTok{::}\NormalTok{gitbook}
\end{Highlighting}
\end{Shaded}

\hypertarget{resources-and-data-access}{%
\chapter{Resources and Data Access}\label{resources-and-data-access}}

\hypertarget{overview-3}{%
\section{Overview}\label{overview-3}}

This section provides information on how to access the datasets used in this project, as well as links to additional resources and references for further reading. Access to some data sources may be restricted due to privacy and ethical considerations. This section also includes links to code repositories and supporting documentation.

\begin{center}\rule{0.5\linewidth}{0.5pt}\end{center}

\hypertarget{data-access}{%
\section{Data Access}\label{data-access}}

The primary datasets used in this project are obtained from official registers provided by Statistics Sweden (SCB) and the National Board of Health and Welfare. Accessing these datasets requires authorization and adherence to data privacy regulations.

\hypertarget{steps-for-data-access}{%
\subsection{Steps for Data Access}\label{steps-for-data-access}}

\begin{enumerate}
\def\labelenumi{\arabic{enumi}.}
\item
  \textbf{Apply for Access}: Researchers interested in accessing the register data should apply through Statistics Sweden or the National Board of Health and Welfare. Application requirements typically include:

  \begin{itemize}
  \tightlist
  \item
    Project description
  \item
    Ethical approval
  \item
    Data protection plan
  \end{itemize}
\item
  \textbf{Data Security and Compliance}: All data access and usage must comply with the General Data Protection Regulation (GDPR) and follow ethical guidelines for handling sensitive personal data.
\item
  \textbf{Contact Information}:

  \begin{itemize}
  \tightlist
  \item
    \textbf{Statistics Sweden}: \href{https://www.scb.se/}{Statistics Sweden's website}
  \item
    \textbf{National Board of Health and Welfare}: \href{https://www.socialstyrelsen.se/}{National Board of Health and Welfare website}
  \end{itemize}
\end{enumerate}

Please refer to each institution's data access policies for specific requirements and further information.

\begin{center}\rule{0.5\linewidth}{0.5pt}\end{center}

\hypertarget{code-repository}{%
\section{Code Repository}\label{code-repository}}

To ensure transparency and reproducibility, the code used for data processing, analysis, and visualization is publicly available in a GitHub repository. You can explore and download the code, and if you are interested in contributing, please feel free to open an issue or submit a pull request.

\begin{itemize}
\tightlist
\item
  \textbf{GitHub Repository}: \href{https://github.com/yourusername/repository-name}{GitHub Link to Code Repository}
\end{itemize}

\hypertarget{contents-of-the-repository}{%
\subsection{Contents of the Repository}\label{contents-of-the-repository}}

The repository includes:
- \textbf{Data Processing Scripts}: Code for data cleaning, transformation, and integration.
- \textbf{Analysis Scripts}: Code used to perform statistical analyses, create derived variables, and generate summary statistics.
- \textbf{Visualization Scripts}: Code for producing charts, graphs, and other visualizations featured in this documentation.

All code is written in R and documented to facilitate understanding and reuse. Please see the README file in the repository for further details on usage and dependencies.

\begin{center}\rule{0.5\linewidth}{0.5pt}\end{center}

\hypertarget{reference-materials}{%
\section{Reference Materials}\label{reference-materials}}

Here are some additional resources and references that provide context for the project and may be helpful for further reading:

\begin{itemize}
\tightlist
\item
  \textbf{Statistics Sweden - Documentation and Reports}: Detailed reports and documentation on Sweden's national register data can be accessed \href{https://www.scb.se/en/documentation/}{here}.
\item
  \textbf{National Board of Health and Welfare - Reports and Statistics}: Publications and statistical reports are available on their official website \href{https://www.socialstyrelsen.se/statistics-and-data/}{here}.
\item
  \textbf{GDPR Compliance}: Information on GDPR and its implications for data processing in research is available on the \href{https://ec.europa.eu/info/law/law-topic/data-protection_en}{European Commission's website}.
\end{itemize}

\begin{center}\rule{0.5\linewidth}{0.5pt}\end{center}

\hypertarget{contact-and-support}{%
\section{Contact and Support}\label{contact-and-support}}

If you have questions about this documentation or need additional support, please reach out to our team:

\begin{itemize}
\tightlist
\item
  \textbf{Email}: \href{mailto:researchteam@example.com}{\nolinkurl{researchteam@example.com}}
\item
  \textbf{Phone}: +46 123 456 789
\item
  \textbf{Address}: Department of Public Health, University of Example, Stockholm, Sweden
\end{itemize}

For additional inquiries regarding data access policies, please contact the respective data providers.

\hypertarget{glossary}{%
\chapter{Glossary}\label{glossary}}

This section provides definitions for key terms used throughout the documentation.

\hypertarget{r-markdown}{%
\section{R Markdown}\label{r-markdown}}

This is an R Markdown document. Markdown is a simple formatting syntax for authoring HTML, PDF, and MS Word documents. For more details on using R Markdown see \url{http://rmarkdown.rstudio.com}.

When you click the \textbf{Knit} button a document will be generated that includes both content as well as the output of any embedded R code chunks within the document. You can embed an R code chunk like this:

\begin{Shaded}
\begin{Highlighting}[]
\FunctionTok{summary}\NormalTok{(cars)}
\end{Highlighting}
\end{Shaded}

\begin{verbatim}
##      speed           dist       
##  Min.   : 4.0   Min.   :  2.00  
##  1st Qu.:12.0   1st Qu.: 26.00  
##  Median :15.0   Median : 36.00  
##  Mean   :15.4   Mean   : 42.98  
##  3rd Qu.:19.0   3rd Qu.: 56.00  
##  Max.   :25.0   Max.   :120.00
\end{verbatim}

\hypertarget{including-plots}{%
\section{Including Plots}\label{including-plots}}

You can also embed plots, for example:

\includegraphics{changelog_files/figure-latex/pressure-1.pdf}

Note that the \texttt{echo\ =\ FALSE} parameter was added to the code chunk to prevent printing of the R code that generated the plot.

  \bibliography{book.bib}

\end{document}
